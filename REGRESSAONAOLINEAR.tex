\documentclass[xcolor=dvipsnames]{beamer}
\usepackage{natbib}
\usepackage{bm} % no preâmbulo
\usepackage{listings}
\usepackage[svgnames]{xcolor}
\usepackage{lstautogobble} % Provides autogobble, which is useful to remove indentation based on first line
\usepackage{graphicx} % Pacote necessário para inserir imagens


\lstset{language=R,
    basicstyle=\small\ttfamily,
    stringstyle=\color{DarkGreen},
    otherkeywords={0,1,2,3,4,5,6,7,8,9},
    morekeywords={TRUE,FALSE},
    deletekeywords={data,frame,length,as,character},
    keywordstyle=\color{blue},
    commentstyle=\color{DarkGreen},
}

\usetheme{Madrid}
\useoutertheme{miniframes} % Alternatively: miniframes, infolines, split
\useinnertheme{circles}

\definecolor{IITHorange}{RGB}{243, 130, 33} % UBC Blue (primary)
\definecolor{IITHyellow}{RGB}{254, 203, 10} % UBC Grey (secondary)

\setbeamercolor{palette primary}{bg=IITHorange,fg=white}
\setbeamercolor{palette secondary}{bg=IITHorange,fg=white}
\setbeamercolor{palette tertiary}{bg=IITHorange,fg=white}
\setbeamercolor{palette quaternary}{bg=IITHorange,fg=white}
\setbeamercolor{structure}{fg=IITHorange} % itemize, enumerate, etc
\setbeamercolor{section in toc}{fg=IITHorange} % TOC sections
\setbeamercolor{alerted text}{fg=orange}

% Override palette coloring with secondary
\setbeamercolor{subsection in head/foot}{bg=IITHyellow,fg=white}

\title[]{Regressão Não Linear}
\date{\today}
\author[Caroline Vasconcelos]
{}
\institute[]{Modelos de Regressão}

\begin{document}
	
	\begin{frame}
		\titlepage
	\end{frame}

        \begin{frame}{Conceito}

        \textit{"Qualquer modelo que não seja linear nos parâmetros desconhecidos é um \textbf{modelo de regressão não linear}"} (Montgomery, 2006).   
        \end{frame}

        \begin{frame}{Conceito}
        Neste caso, pode-se seguir três caminhos diferentes:
        \begin{enumerate}
            \item  linearizar a relação transformando os dados;
            \item ajustar modelos polinomiais ou splines complexos aos dados;
            \item \textbf{ajustar funções não lineares aos dados.}
        \end{enumerate}
            
        \end{frame}

         \begin{frame}{Exemplo}
         O modelo

       \alert{$$\textbf{y} =\theta_{1}e^{\theta_{2}x}+\epsilon$$}

        é não linear nos parâmetros desconhecidos \alert{$\theta_{1}$} e \alert{$\theta_{2}$}. Podendo ser escrito também como

        \alert{$$\textbf{y} = f(\textbf{x},\bm{\theta}) + \epsilon$$}
            
        \end{frame}

        \begin{frame}
  \frametitle{Conceito}

  Nos modelos de regressão não linear pelo menos uma das derivadas da função esperança $f(\textbf{x},\bm{\theta})$ depende de um dos parâmetros. Na regressão linear essas derivadas não são função de tais parâmetros desconhecidos.
  
  \begin{columns}[T] % alinhamento no topo
    \begin{column}{0.475\textwidth} % largura da primeira coluna
      \begin{block}{Regressão Linear}
      Modelo:
      $$y = \beta_{0}+\beta_{1}x_{1}+\beta_{2}x_{2}+...+\beta_{k}x_{k}+\epsilon$$\\

      Função esperança:\\
      $$f(\textbf{x},\bm{\beta})$$
          
      Derivada: $\frac{\partial f(\textbf{x},\bm{\beta})}{\partial \beta_{j}} = x_{j}$
       
        
      \end{block}
    \end{column}
    
    \begin{column}{0.475\textwidth} % largura da segunda coluna
      \begin{block}{Regressão Não Linear}
        Modelo:
        $$y = \theta_{1}e^{\theta_{2}x}+\epsilon$$
        
        Função esperança:\\
        $$f(\textbf{x},\bm{\theta})$$

        Derivadas: \\
         $\frac{\partial f(\textbf{x},\bm{\theta})}{\partial \theta_{1}} = e^{\theta_{2}x}$;
         $\frac{\partial f(\textbf{x},\bm{\theta})}{\partial \theta_{2}} = \theta_{1}xe^{\theta_{2}x}$   
        
      \end{block}
    \end{column}
  \end{columns}
  
\end{frame}


       \begin{frame}{Vantagens}
       De maneira resumida, os modelos não lineares (MNL) têm as seguintes vantagens sobre os modelos lineares (ML):
         \begin{enumerate}
             \item Sua escolha têm sustentação baseada em teoria ou princípios mecanísticos (físicos, químicos ou biológicos) ou qualquer outra informação prévia;
             \item Certos parâmetros são quantidade de interesse para o pesquisador providos de interpretação;
             \item São parcimoniosos pois tipicamente possuem menos parâmetros;
             \item Partem do conhecimento do pesquisador sobre o fenômeno alvo.

         \end{enumerate}
           
       \end{frame}

       \begin{frame}{Desvantagens}
          \begin{enumerate}
              \item Requerem procedimentos iterativos de estimação baseados no fornecimento de valores iniciais para os parâmetros;
              \item Métodos de inferência são aproximados;
              \item Exigem conhecimento do pesquisador sobre o fenômeno alvo.
          \end{enumerate}
           
       \end{frame}

       \begin{frame}{Modelos de Regressão Não Linear: nota}

      \textbf{Idealmente} um modelo de regressão não linear é escolhido com base em considerações teóricas sobre o problema, o que às vezes faz com que eles sejam usados em situações muito específicas.

      \vspace{0.5cm}
      Nesse caso, as aplicações mais comuns estão relacionadas a \textbf{modelos de crescimento}, tipicamente na área da \textbf{biologia} onde plantas e organismo crescem com o tempo.
           
       \end{frame}

       \begin{frame}{Estudo de caso: uma curva de degradação}

       Um solo foi enriquecido com o herbicida metramitron até a concentração de 100 ng/g. Foi então colocado em 24 recipientes de alumínio, dentro de uma câmara climática a 20°C. Foram selecionados 3 recipientes aleatoriamente em 8 momentos diferentes e a concentração residual de metamitron foi medida.
           
       \end{frame}

 
        \begin{frame}[fragile]{Dados}
           \begin{lstlisting}[language=R]
x <- "https://www.casaonofri.it/_datasets/degradation.csv"
dataset <- read.csv(x, header=T)
head(dataset, 10)
##    Time   Conc
## 1     0  96.40
## 2    10  46.30
## 3    20  21.20
## 4    30  17.89
## 5    40  10.10
## 6    50   6.90
## 7    60   3.50
## 8    70   1.90
## 9     0 102.30
## 10   10  49.20
           \end{lstlisting}
        \end{frame}

        \begin{frame}{Gráfico de dispersão}
           \begin{figure}[h]
          
           \includegraphics[height=6cm]{image1.png}
            \caption{Degradação do metamitron no solo.}
    
          \end{figure}
        \end{frame}

        \begin{frame}{Formas de funções}
        
           \begin{figure}[h]
          
           \includegraphics[height=6cm]{image2.png}
            \caption{Formas das funções mais importantes.}
    
          \end{figure}
        \end{frame}

         \begin{frame}{Equações por trás das funções}
        
           \begin{figure}[h]
          
           \includegraphics[height=6cm]{image3.png}
            \caption{Equações e funções no R.}
    
          \end{figure}
        \end{frame}

         \begin{frame}{Modelo}

        \textit{$$Y_{i} = ae^{-kX_i}+\epsilon_i$$}

       Onde \\
       \textit{Y} é a \textbf{concentração no tempo} \textit{X} \\
       \textit{a} é a \textbf{concentração inicial de metamitrona} \\
       \textit{k} é a t\textbf{axa de degradação constante}.
             
         \end{frame}

        \begin{frame}[fragile]{Estimação dos parâmetros: linearização}
           \begin{lstlisting}[language=R]
mod <- lm(log(Conc) ~ Time, data=dataset)
summary(mod)
## Call:
## lm(formula = log(Conc) ~ Time, data = dataset)
## Residuals:
##      Min       1Q   Median       3Q      Max 
## -2.11738 -0.09583  0.05336  0.31166  1.01243 

## Coefficients:
##              Estimate Std. Error t value Pr(>|t|)    
## (Intercept)  4.662874   0.257325   18.12 1.04e-14 ***
## Time        -0.071906   0.006151  -11.69 6.56e-11 ***
## Residual standard error: 0.6905 on 22 degrees of freedom
## Multiple R-squared:  0.8613, Adjusted R-squared:  0.855 
## F-statistic: 136.6 on 1 and 22 DF,  p-value: 6.564e-11
           \end{lstlisting}
        \end{frame} 
        
        \begin{frame}[fragile]{Estimação dos parâmetros: linearização}
           \begin{lstlisting}[language=R]
par(mfrow = c(1,2))
plot(mod, which = 1)
plot(mod, which = 2)
           \end{lstlisting}
           \begin{figure}[h]
           \includegraphics[height=5.5cm]{image4.png}        
          \end{figure}          
        \end{frame}

         \begin{frame}[fragile]{Estimação dos parâmetros: aproximação polinomial}
           \begin{lstlisting}[language=R]
mod2 <- lm(Conc ~ Time + I(Time^2), data=dataset)
pred <- predict(mod2, newdata = data.frame(Time = seq(0, 70, by = 0.1)))
plot(Conc ~ Time, data=dataset)
lines(pred ~ seq(0, 70, by = 0.1), col = "red")
           \end{lstlisting}
           \begin{figure}[h]
           \includegraphics[height=5cm]{image5.png}        
          \end{figure} 
           
        \end{frame}

         \begin{frame}[fragile]{Estimação dos parâmetros: mínimos quadrados não lineares}
           \begin{lstlisting}[language=R]
modNlin <- nls(Conc ~ A*exp(-k*Time), 
               start=list(A=100, k=0.05), 
               data = dataset)
summary(modNlin)

## Formula: Conc ~ A * exp(-k * Time)

## Parameters:
##    Estimate Std. Error t value Pr(>|t|)    
## A 99.634902   1.461047   68.19   <2e-16 ***
## k  0.067039   0.001887   35.53   <2e-16 ***
## 
## Residual standard error: 2.621 on 22 degrees of freedom
## 
## Number of iterations to convergence: 5 
## Achieved convergence tolerance: 4.33e-07
           \end{lstlisting}
        \end{frame}

        \begin{frame}[fragile]{Ajuste do modelo}
           \begin{lstlisting}[language=R]
par(mfrow=c(1,2))
plot(modNlin, which = 1)
plot(modNlin, which = 2)
           \end{lstlisting}
           \begin{figure}[h]
           \includegraphics[height=5cm]{image6.png}        
          \end{figure}
           
        \end{frame}

         \begin{frame}[fragile]{Ajuste do modelo}
           \begin{lstlisting}[language=R]
plot(modNlin, type = "means",
        xlab = "Time(d)", ylab = "Concentration(ng/g)")
           \end{lstlisting}
           \begin{figure}[h]
           \includegraphics[height=6cm]{image7.png}        
          \end{figure}
           
        \end{frame}
                   
          \begin{frame}[fragile]{Medida de qualidade do ajuste}
           \begin{lstlisting}[language=R]
MSE <- summary(modNlin)$sigma ^ 2
MST <- var(dataset$Conc)
1 - MSE/MST
## [1] 0.9936359
           \end{lstlisting}
        \end{frame}

        \begin{frame}[fragile]{Predição}
           \begin{lstlisting}[language=R]
func <- list(~A * exp(-k * time))
const <- data.frame(time = c(5, 10, 15)) 
gnlht(modNlin, func,  const)
##         Form          time Estimate     p-value
## 1 A * exp(-k * time)    5 71.25873  5.340741e-28
## 2 A * exp(-k * time)   10 50.96413  3.163518e-25
## 3 A * exp(-k * time)   15 36.44947  6.029672e-22
           \end{lstlisting}         
        \end{frame}

         \begin{frame}{Referências}
         
MONTGOMERY, Douglas C.; PECK, Elizabeth A.; VINING, G. Geoffrey. \textbf{Introduction to Linear Regression Analysis}. 4ª ed. Hoboken: John Wiley & Sons, 2006.\\

\vspace{0.5cm}
NABERGOI, Carlos. \textbf{Curso de Modelos Não Lineares em R}. Disponível em: https://www.ime.unicamp.br/~cnaber/cursomodelosnaolinearesR.pdf. Acesso em: 21 jul. 2024.\\

\vspace{0.5cm}
STATFORBIOLOGY.COM. \textbf{Nonlinear Regression}. Disponível em: https://www.statforbiology.com/-statbookeng/nonlinear-regression. Acesso em: 21 jul. 2024.        
         \end{frame}

         \begin{frame}[fragile]{}

    \hspace{2cm}     Obrigada! :)
    \vspace{1.5cm}
           \begin{lstlisting}[language=R]
               print("sextou")
           \end{lstlisting}
        \end{frame}

      
	
	

        
\end{document}